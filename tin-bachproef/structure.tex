%%=============================================================================
%% LaTeX sjabloon voor de bachelorproef, HoGent Bedrijf en Organisatie
%% Opleiding toegepaste informatica
%%
%% Structuur en algemene vormgeving
%%
%% Auteur: Bert Van Vreckem <bert.vanvreckem@hogent.be>
%% Licentie: CC-BY https://creativecommons.org/licenses/by/4.0/
%%=============================================================================

%%---------- Packages ---------------------------------------------------------

\usepackage[utf8]{inputenc}  % Accenten gebruiken in tekst (vb. é ipv \'e)
\usepackage{amsfonts}        % AMS math packages: extra wiskundige
\usepackage{amsmath}         %   symbolen (o.a. getallen-
\usepackage{amssymb}         %   verzamelingen N, R, Z, Q, etc.)
\usepackage[english,dutch]{babel}    % Taalinstellingen: woordsplitsingen,
                             %  commando's voor speciale karakters
                             %  ("dutch" voor NL)
\usepackage{iflang}
\usepackage{eurosym}         % Euro-symbool €
\usepackage{geometry}
\usepackage{graphicx}        % Invoegen van tekeningen
\usepackage[pdftex,bookmarks=true]{hyperref}
                             % PDF krijgt klikbare links & verwijzingen,
                             %  inhoudstafel
\usepackage{listings}        % Broncode mooi opmaken
\usepackage{multirow}        % Tekst over verschillende cellen in tabellen
\usepackage{rotating}        % Tabellen en figuren roteren
\usepackage{natbib}          % Betere bibliografiestijlen
\usepackage{fancyhdr}        % Pagina-opmaak met hoofd- en voettekst

\usepackage[T1]{fontenc}     % Ivm lettertypes
\usepackage{lmodern}
\usepackage{textcomp}

\usepackage{lipsum}          % Voor vultekst (lorem ipsum)

%%---------- Layout -----------------------------------------------------------

\pagestyle{fancy}            % Hoofdingen invoegen
\renewcommand{\sectionmark}[1]{} % enkel hoofdstuktitel in hoofding, geen
                             % sectietitel (vermijd overlap)

\newcommand{\HRule}{\rule{\linewidth}{0.5mm}}
                             % horizontale lijn (voor titelpagina)

% Leeg blad
\newcommand{\emptypage}{
\newpage
\thispagestyle{empty}
\mbox{}
\newpage
}

% Gebruik een schreefloos lettertype ipv het "oubollig" uitziende
% Computer Modern
\renewcommand{\familydefault}{\sfdefault}

% Commando voor invoegen Java-broncodebestanden (dank aan Niels Corneille)
% Gebruik: \codefragment{source/MijnKlasse.java}{Uitleg bij de code}
\newcommand{\codefragment}[2]{ \lstset{%
  language=java,
  breaklines=true,
  float=th,
  caption={#2},
  basicstyle=\scriptsize,
  frame=single,
  extendedchars=\true
}
\lstinputlisting{#1}}

%%---------- Voorblad ---------------------------------------------------------

\newcommand{\inserttitlepage}{%
\begin{titlepage}
  \newgeometry{top=2cm,bottom=1.5cm,left=1.5cm,right=1.5cm}
  \begin{center}

    \begingroup
    \rmfamily
    \includegraphics[width=2.5cm]{img/HG-beeldmerk-woordmerk}\\[.5cm]
    \IfLanguageName{dutch}{Faculteit Bedrijf en Organisatie}{Faculty of Business and Information Management}\\[3cm]
    \titel
    \vfill
    \student\\[3.5cm]
    \IfLanguageName{dutch}{Scriptie voorgedragen tot het bekomen van de graad van\\Professionele bachelor in de toegepaste informatica}{Thesis submitted in partial fulfillment of the requirements for the degree of\\Professional bachelor of applied computer science}\\[2cm]
    Promotor:\\
    \promotor\\
    Co-promotor:\\
    \copromotor\\[2.5cm]
    \IfLanguageName{dutch}{Instelling}{Institution}: \instelling\\[.5cm]
    \IfLanguageName{dutch}{Academiejaar}{Academic year}: \academiejaar\\[.5cm]
    \IfLanguageName{dutch}{%
    \ifcase \examenperiode \or Eerste \or Tweede \else Derde \fi examenperiode}{%
    \ifcase \examenperiode \or First \or Second \else Third \fi examination period}
    \endgroup

  \end{center}
  \restoregeometry
\end{titlepage}
  \emptypage
\begin{titlepage}
  \newgeometry{top=5.35cm,bottom=1.5cm,left=1.5cm,right=1.5cm}
  \begin{center}

    \begingroup
    \rmfamily
    \IfLanguageName{dutch}{Faculteit Bedrijf en Organisatie}{Faculty of Business and Information Management}\\[3cm]
    \titel
    \vfill
    \student\\[3.5cm]
    \IfLanguageName{dutch}{Scriptie voorgedragen tot het bekomen van de graad van\\Professionele bachelor in de toegepaste informatica}{Thesis submitted in partial fulfillment of the requirements for the degree of\\Professional bachelor of applied computer science}\\[2cm]
    Promotor:\\
    \promotor\\
    Co-promotor:\\
    \copromotor\\[2.5cm]
    Instelling: \instelling\\[.5cm]
    Academiejaar: \academiejaar\\[.5cm]
    \examenperiode
    \endgroup

  \end{center}
  \restoregeometry
\end{titlepage}
}

